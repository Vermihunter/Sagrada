\chapter*{Introduction}
\addcontentsline{toc}{chapter}{Introduction}


In recent years, the application of artificial intelligence (AI) to traditional board games 
has reached unprecedented heights, showcasing remarkable advancements in strategic gameplay. 
Games such as Go, chess, and poker, each with their unique challenges, have served as testing 
grounds for AI algorithms. Go, known for its large branching factor and complex strategies, 
has been revolutionized by AI, notably with AlphaGo's victories against world champions. 

Sagrada is a dice-drafting board game designed by Adrian Adamescu and Daryl Andrews and published in 2017.
Each player constructs a stained-glass window using dice on a personal rectangular 4×5 board with restrictions 
on the types of dice that can be played on each space. Players gain points by completing public and secret objectives 
for dice placements, and the one with the most points after ten rounds is the winner. In case of a draw in total points
achieved, the tie is broken by other criteria implying that there is always a winner in every game.

Unlike Go and chess, which are deterministic games with no hidden information, Sagrada includes both randomness and hidden information, 
adding layers of complexity to strategic decision-making. Additionally, Sagrada has a high branching factor, 
making its gameplay significantly complex. the motivation behind this study lies in solving the challenges of imperfect information and 
high branching factor in developing a strong AI player for Sagrada.

Prior to the composition of this bachelor's thesis, a digital adaptation of the board game Sagrada had already been released 
by Dire Wolf Digital. There is no public documentation available about the AI players of their implementation.

The primary aim of this study is to evaluate and compare the performance of different AI players in 
my implementation of the digital adaptation of Sagrada. My goal was to implement and analyze various AI strategies, 
including ones such as Minimax, Monte Carlo Tree Search (MCTS), and rules-based agents. This research aims to 
examine the effectiveness of each approach by experimenting with different configurations of the agents.


Through statistical analysis and gameplay simulations, this thesis seeks to provide 
insights into the strengths and weaknesses of AI players, offering valuable implications for the 
development of intelligent gaming agents. 

This thesis will concentrate on the two-player version of the Sagrada board game, although it's designed for 1 to 4 players.